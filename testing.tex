%%%%%%%%%%%%%%%%%%%%%%%%%%%%%%%%%%%%%%%%%%%%%%%%%%%%%%%%%%%%%%%%%%%%%%%%%%%%%%%%
\chapter{ТЕСТИРОВАНИЕ ПРОТОТИПА И АНАЛИЗ ПОЛУЧЕННЫХ РЕЗУЛЬТАТОВ}
\label{chapter:testing}
%%%%%%%%%%%%%%%%%%%%%%%%%%%%%%%%%%%%%%%%%%%%%%%%%%%%%%%%%%%%%%%%%%%%%%%%%%%%%%%%
В данном разделе проводится исследование показателей разработанной технологии путем анализа результатов запуска прототипа на нескольких реальных программных проектах. Исследованию подвергается работоспособность прототипа.

%%%%%%%%%%%%%%%%%%%%%%%%%%%%%%%%%%%%%%%%%%%%%%%%%%%%%%%%%%%%%%%%%%%%%%%%%%%%%%%%
\section{Описание тестовых проектов}
%%%%%%%%%%%%%%%%%%%%%%%%%%%%%%%%%%%%%%%%%%%%%%%%%%%%%%%%%%%%%%%%%%%%%%%%%%%%%%%%
Для тестирования прототипа было выбрано несколько проектов с открытым исходным кодом:
\begin{itemize}
\item библиотека beanstalkd --- простая и высокопроизводительная очередь, разработанная для уменьшения времени отклика при обращении к веб-сервисам большого объема путем асинхронного запуска сложных задач;
\item библиотека iputils --- набор небольших утилит для работы с сетью в Lunix;
\item sslsplit --- инструмент для создания атак <<человек посередине>>(Man in the middle, MITM) в сети с SSL/TSL шифрованием (предназначен для проверки безопасности приложений и уязвимости сети);
\item git --- распределенная СКВ;
\item vim ---  многофункциональный свободный текстовый редактор с полной свободой настройки и автоматизации, обладающий большим количеством расширений и надстроек.
\end{itemize}

Данные проекты достаточно сильно отличаются друг от друга как по объему, так и по структуре. Проект beanstalkd является достаточно маленьким ($SLOC \approx 7.5k$), весь код является сильносвязным, его нельзя разбить на отдельные модули. Проект iputils также является маленьким ($SLOC \approx 12k$), однако он состоит из 11 еще более маленьких подпроектов.

SslSplit является проектом среднего размера ($SLOC \approx 100k$), в проекте отсутствует декомпозиция, компилируется в один исполняемый файл.

Проекты vim и git представляют из себя большие программные проекты ($SLOC \approx 330k$ в каждом). Git содержит в себе несколько подсистем, одна из которых является большой, а остальные сравнительно маленькими. Vim является сильносвязным и компилируется в один объемный исполняемый файл.

Данный тестовый набор позволит проверить эффективность разработанной технологии как для малых, так и для больших проектов с разной степенью функциональной декомпозиции.

%%%%%%%%%%%%%%%%%%%%%%%%%%%%%%%%%%%%%%%%%%%%%%%%%%%%%%%%%%%%%%%%%%%%%%%%%%%%%%%%
\section{Схема работы прототипа}
%%%%%%%%%%%%%%%%%%%%%%%%%%%%%%%%%%%%%%%%%%%%%%%%%%%%%%%%%%%%%%%%%%%%%%%%%%%%%%%%
Общая схема работы прототипа в тестовом режиме выглядит следующим образом:
\begin{itemize}
\item для целевого проекта из СКВ получается его последняя версия;
\item в файле конфигурации проекта в качестве используемого компилятора указывается Borealis;
\item производится сборка проекта;
\item производится анализ результатов работы прототипа.
\end{itemize}

%%%%%%%%%%%%%%%%%%%%%%%%%%%%%%%%%%%%%%%%%%%%%%%%%%%%%%%%%%%%%%%%%%%%%%%%%%%%%%%%
\section{Конфигурация прототипа для выполнения тестирования}
%%%%%%%%%%%%%%%%%%%%%%%%%%%%%%%%%%%%%%%%%%%%%%%%%%%%%%%%%%%%%%%%%%%%%%%%%%%%%%%%
Для проверки работы прототипа на выбранном тестовом наборе изначально была собрана объемная база данных предикатов. Сбор базы производился на случайной выборке 40 проектов различного объема и сложности. Выборка производилась вручную. В результате этого была собрана база данных предикатов объемом $\sim{80}$Мб.

Так же, для проведения анализа проектов необходимо выбрать значения констант $X$ и $K$, которые определяют параметры алгоритма (см. раздел \ref{section:merging}). В данной работе главной целью является точность результатов анализа, но полнота так же важна. Поэтому, было принято решение выбрать значения этих коэффициентов с небольшим смещением в сторону точности: $K=0.68$, $X=3$.

%%%%%%%%%%%%%%%%%%%%%%%%%%%%%%%%%%%%%%%%%%%%%%%%%%%%%%%%%%%%%%%%%%%%%%%%%%%%%%%%
\section{Результаты тестирования}
%%%%%%%%%%%%%%%%%%%%%%%%%%%%%%%%%%%%%%%%%%%%%%%%%%%%%%%%%%%%%%%%%%%%%%%%%%%%%%%%
Результаты анализа тестовых проектов приведены в таблице \ref{table:testing}.
\begin{table}
	\caption{Результаты тестирования}
	\begin{center}
	\begin{tabular}{|l|l|l|}
	\hline 
	\textbf{Проект} & \textbf{SLOC} & \textbf{Кол-во предусловий}	\\ 
	\hline 
	beanstalkd & $7.5k$ & 10 \\ 
	\hline 
	iputils & $12k$ & 15  \\ 
	\hline 
	sslsplit & $100k$  & 65\\ 
	\hline 
	vim & $320k$ & - \\ 
	\hline 
	git & $340k$ & - \\ 
	\hline 
	\end{tabular} 
	\end{center}
	\label{table:testing}
\end{table}

Примеры извлеченных предикатов приведены на листинге \ref{listings:extractionExample}. Полный список найденных контрактов приведен в приложении \ref{app:results}.

\lstinputlisting[
label={listings:extractionExample},
caption={Примеры контрактов, извлеченных для проекта iputils},
]
{src/extractionExample}

По результатам тестирования можно сделать вывод о том, что разработанный прототип показал свою работоспособность. На тестовом наборе проектов было извлечено большое количество предусловий.

%%%%%%%%%%%%%%%%%%%%%%%%%%%%%%%%%%%%%%%%%%%%%%%%%%%%%%%%%%%%%%%%%%%%%%%%%%%%%%%%
\section{Резюме}
%%%%%%%%%%%%%%%%%%%%%%%%%%%%%%%%%%%%%%%%%%%%%%%%%%%%%%%%%%%%%%%%%%%%%%%%%%%%%%%%
В данном разделе выполнено тестирование разработанного прототипа системы автоматического извлечения контрактов из исходного кода. Показана его работоспособность. Анализ результатов тестирования говорит о целесообразности применения предлагаемой в данной работе технологии автоматического извлечения контрактов из исходного кода.
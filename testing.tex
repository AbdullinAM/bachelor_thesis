%%%%%%%%%%%%%%%%%%%%%%%%%%%%%%%%%%%%%%%%%%%%%%%%%%%%%%%%%%%%%%%%%%%%%%%%%%%%%%%%
\chapter{ТЕСТИРОВАНИЕ ПРОТОТИПА И АНАЛИЗ ПОЛУЧЕННЫХ РЕЗУЛЬТАТОВ}
\label{chapter:testing}
%%%%%%%%%%%%%%%%%%%%%%%%%%%%%%%%%%%%%%%%%%%%%%%%%%%%%%%%%%%%%%%%%%%%%%%%%%%%%%%%
В данном разделе проводится исследование показателей разработанной технологии путем анализа результатов запуска прототипа на нескольких реальных программных проектах. Исследованию подвергается эффективность алгоритма (качество и количество извлеченных контрактов).

%%%%%%%%%%%%%%%%%%%%%%%%%%%%%%%%%%%%%%%%%%%%%%%%%%%%%%%%%%%%%%%%%%%%%%%%%%%%%%%%
\section{Схема работы прототипа}
%%%%%%%%%%%%%%%%%%%%%%%%%%%%%%%%%%%%%%%%%%%%%%%%%%%%%%%%%%%%%%%%%%%%%%%%%%%%%%%%
Общая схема работы прототипа в тестовом режиме выглядит следующим образом:
\begin{itemize}
\item для целевого проекта из СКВ получается его последняя версия;
\item в файле конфигурации проекта в качестве используемого компилятора указывается Borealis;
\item производится сборка проекта;
\item производится анализ результатов работы прототипа. Анализ представляет собой ручную проверку корректности полученных контрактов.
\end{itemize}

%%%%%%%%%%%%%%%%%%%%%%%%%%%%%%%%%%%%%%%%%%%%%%%%%%%%%%%%%%%%%%%%%%%%%%%%%%%%%%%%
\section{Описание тестовых проектов}
%%%%%%%%%%%%%%%%%%%%%%%%%%%%%%%%%%%%%%%%%%%%%%%%%%%%%%%%%%%%%%%%%%%%%%%%%%%%%%%%
Для тестирования прототипа было выбрано несколько проектов с открытым исходным кодом:
\begin{itemize}
\item библиотека beanstalkd --- простая и высокопроизводительная очередь, разработанная для уменьшения времени отклика при обращении к веб-сервисам большого объема путем асинхронного запуска сложных задач;
\item библиотека iputils --- набор небольших утилит для работы с сетью в Lunix;
\item sslsplit --- инструмент для создания атак <<человек посередине>>(Man in the middle, MITM) в сети с SSL/TSL шифрованием (предназначен для проверки безопасности приложений и уязвимости сети);
\item git --- распределенная СКВ;
\item vim ---  мощнейший свободный текстовый редактор с полной свободой настройки и автоматизации, обладающий большим количеством расширений и надстроек.
\end{itemize}

Данные проекты достаточно сильно отличаются друг от друга как по объему, так и по структуре. Проект beanstalkd является достаточно маленьким ($SLOC \approx 7.5k$), весь код является сильносвязным, его нельзя разбить на отдельные модули. Проект iputils также является маленьким ($SLOC \approx 12k$), однако он состоит из 11 еще более маленьких подпроектов.

SslSplit является проектом среднего размера ($SLOC \approx 100k$), в проекте отсутствует декомпозиция, компилируется в один исполняемый файл.

Проекты vim и git представляют из себя большие программные проекты ($SLOC \approx 340k$ в каждом). Git содержит в себе несколько подсистем, одна из которых является большой, а остальные сравнительно маленькими. Vim является сильносвязным и компилируется в один объемный исполняемый файл.

Данный тестовый набор позволит проверить эффективность разработанной технологии как для малых, так и для больших проектов с разной степенью функциональной декомпозиции.

%%%%%%%%%%%%%%%%%%%%%%%%%%%%%%%%%%%%%%%%%%%%%%%%%%%%%%%%%%%%%%%%%%%%%%%%%%%%%%%%
\section{Конфигурация прототипа для выполнения тестирования}
%%%%%%%%%%%%%%%%%%%%%%%%%%%%%%%%%%%%%%%%%%%%%%%%%%%%%%%%%%%%%%%%%%%%%%%%%%%%%%%%
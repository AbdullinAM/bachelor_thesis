%%%%%%%%%%%%%%%%%%%%%%%%%%%%%%%%%%%%%%%%%%%%%%%%%%%%%%%%%%%%%%%%%%%%%%%%%%%%%%%%
\intro
%%%%%%%%%%%%%%%%%%%%%%%%%%%%%%%%%%%%%%%%%%%%%%%%%%%%%%%%%%%%%%%%%%%%%%%%%%%%%%%%
В настоящее время анализ программ становится все более важным, так как это один из основных способов проверки и обеспечения качества программного обеспечения. В современном мире анализ программ сталкивается с множеством проблем, наиболее важной из которых является необходимость повышения полноты и точности анализа. Полнота анализа --- это отношение числа корректно найденных дефектов к общему числу дефектов в программе. Точность --- это отношение числа корректно найденных дефектов к общему числу найденных дефектов. Корректно найденными являются дефекты, которые действительно присутствуют в анализируемой программе.

Различные виды анализа решают эту проблему разными способами. Классическим подходом к решению данной проблемы в статическом анализе является использование пользовательских спецификаций (контрактов). Пользовательские спецификации --- это описания поведения программы, написанные пользователем. Спецификации дают анализу дополнительную информацию, которая позволяет находить дефекты, связанные с особенностями использования программы или ее компонентов. Использование подобного подхода связано с определенными проблемами, основной из которых является необходимость ручного написания данных спецификаций. Альтернативным подходом, рассматриваемым в данной работе, является автоматическое извлечение контрактов из исходного кода.

Основная идея данного подхода заключается в анализе исходного кода с целью выявления шаблонов вызова различных функций, на основе которых далее формируются контракты. Целью данной работы является разработка алгоритма автоматического извлечения контрактов для улучшения полноты и точности статического анализа. В рамках работы также планируется разработка модуля автоматического извлечения контрактов для системы статического анализа Borealis~\cite{borealis}.

Работа состоит из 5 разделов. В разделе 1 рассматривается использование контрактов в статическом анализе программ, существующие исследования в области автоматического извлечения контрактов. Проводится анализ существующих алгоритмов, определяются их основные особенности.

Раздел 2 посвящен постановке задачи автоматического извлечения контрактов. Выбирается целевой язык программирования, ключевые параметры технологии. В этом разделе также ставится задача разработки модуля автоматического извлечения контрактов для системы статического анализа Borealis.

В 3 разделе рассматривается модель представления кода и описывается предлагаемый алгоритм извлечения контрактов вместе с его ограничениями в рамках выбранной модели представления кода.

Раздел 4 посвящен разработке модуля автоматического извлечения контрактов для системы Borealis. В разделе рассматривается архитектура разрабатываемого прототипа, описываются выбранные инструменты и библиотеки, примененные при разработке.

В разделе 5 представлены результаты тестирования созданного модуля. Показывается применимость данного подхода на реальных программных проектах.

%%%%%%%%%%%%%%%%%%%%%%%%%%%%%%%%%%%%%%%%%%%%%%%%%%%%%%%%%%%%%%%%%%%%%%%%%%%%%%%%
\intro
%%%%%%%%%%%%%%%%%%%%%%%%%%%%%%%%%%%%%%%%%%%%%%%%%%%%%%%%%%%%%%%%%%%%%%%%%%%%%%%%
В настоящее время анализ программ становится все более важным, так как один из основных способов проверки качества программного обеспечения. В современном мире анализ программ сталкивается с множеством проблем.  Наиболее важными проблемами являются способы повышения полноты и точности анализа.

Классическим подходом к решения данной проблемы в статическом анализе является использование пользовательских спецификаций (контрактов). Использование подобного подхода так же связано с определенными проблемами. Альтернативным подходом, рассматриваемым в данной работе является автоматическое извлечение контрактов из исходного кода.

Основной идеей данного подхода является анализ большого количества корректного исходного кода с целью выявления шаблонов вызова различных функций. Целью данной работы является разработка алгоритма автоматического извлечения контрактов для улучшения полноты и точности статического анализа. В рамках работы так же планируется разработка модуля автоматического извлечения контрактов для системы статического анализа Borealis[вставить ссылку].

Работа состоит из 5 разделов. В разделе 1 рассматриваются существующие исследования в области автоматического извлечения контрактов. Проводится анализ существующих алгоритмов, определяются их основные особенности.

Раздел 2 посвящен постановке задачи автоматического извлечения контрактов. Выбирается целевой язык программирования, выбираются ключевые параметры технологии. В этом разделе также ставится задача разработки модуля автоматического извлечения контрактов для системы статического анализа Borealis.

В 3 разделе описывается предлагаемый алгоритм извлечения контрактов. Рассматривается модель представления кода. Рассматриваются особенности алгоритма для описанной модели.

4 раздел посвящен разработке модуля автоматического извлечения контрактов для системы Borealis. Рассматривается структура системы, ее особенности. Описываются особенности реализации созданного алгоритма.

В разделе 5 представлены результаты тестирования созданного модуля. Показывается эффективность описанного подхода. Определяются недостатки описанной технологии и способы их преодоления.

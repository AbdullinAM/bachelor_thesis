%%%%%%%%%%%%%%%%%%%%%%%%%%%%%%%%%%%%%%%%%%%%%%%%%%%%%%%%%%%%%%%%%%%%%%%%%%%%%%%%
\intro
%%%%%%%%%%%%%%%%%%%%%%%%%%%%%%%%%%%%%%%%%%%%%%%%%%%%%%%%%%%%%%%%%%%%%%%%%%%%%%%%
В настоящее время анализ программ становится все более важным, так как это один из основных способов проверки качества программного обеспечения. В современном мире анализ программ сталкивается с множеством проблем.  Наиболее важной проблемой является необходимость повышения полноты и точности анализа.

Различные виды анализа решают эту проблему разными способами. Классическим подходом к решению данной проблемы в статическом анализе является использование пользовательских спецификаций (контрактов). Использование подобного подхода так же связано с определенными проблемами. Альтернативным подходом, рассматриваемым в данной работе является автоматическое извлечение контрактов из исходного кода.

Основной идеей данного подхода является анализ большого количества корректного исходного кода с целью выявления шаблонов вызова различных функций. Целью данной работы является разработка алгоритма автоматического извлечения контрактов для улучшения полноты и точности статического анализа. В рамках работы так же планируется разработка модуля автоматического извлечения контрактов для системы статического анализа Borealis\cite{borealis}.

Работа состоит из 5 разделов. В разделе 1 рассматривается предметная область контрактного программирования. Рассматривается использование контрактов в статическом анализе программ. Рассматриваются существующие исследования в области автоматического извлечения контрактов. Проводится анализ существующих алгоритмов, определяются их основные особенности.

Раздел 2 посвящен постановке задачи автоматического извлечения контрактов. Выбирается целевой язык программирования, выбираются ключевые параметры технологии. В этом разделе также ставится задача разработки модуля автоматического извлечения контрактов для системы статического анализа Borealis.

В 3 разделе рассматривается модель представления кода и описывается предлагаемый алгоритм извлечения контрактов. Описываются особенности алгоритма для выбранной модели представления кода.

4 раздел посвящен разработке модуля автоматического извлечения контрактов для системы Borealis. В разделе рассматривается архитектура разрабатываемого прототипа, описываются выбранные инструменты и библиотеки, примененные при разработке.

В разделе 5 представлены результаты тестирования созданного модуля. Показывается применимость данного подхода на реальных программных проектах.

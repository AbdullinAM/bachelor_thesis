%%%%%%%%%%%%%%%%%%%%%%%%%%%%%%%%%%%%%%%%%%%%%%%%%%%%%%%%%%%%%%%%%%%%%%%%%%%%%%%%
\conclusion
%%%%%%%%%%%%%%%%%%%%%%%%%%%%%%%%%%%%%%%%%%%%%%%%%%%%%%%%%%%%%%%%%%%%%%%%%%%%%%%%
Одной из главных проблем статического анализа является необходимость повышения его полноты и точности. Одним из способов решения данной проблемы является ручное задание пользовательских спецификаций к исходному коду. В данной работе рассматривается другой подход к решению данной проблемы: технология автоматического извлечения контрактов функций из исходного кода программ.

В работе выполнен анализ предметной области, показано применение контрактов в современном мире. Также в работе выполнен анализ исследований в области автоматического извлечения контрактов из исходного кода. Показана актуальность разработки технологии статического анализа программ с целью автоматического извлечения контрактов.

Основой предложенного подхода является анализ большого количества программных проектов с целью извлечения шаблонов вызовов функций в этих проектах. В работе представлен полный алгоритм автоматического извлечения контрактов из исходного кода.

На базе полученной технологии разработан прототип системы автоматического извлечения контрактов. Прототип состоит из двух частей: модуль автоматического извлечения контрактов для системы статического анализа Borealis и модуль хранения результатов анализа, взаимодействующий со встраиваемым хранилищем. Разработанный прототип был протестирован на наборе реальных программных проектов различной сложности. Тестирование показало целесообразность применения технологии для автоматического извлечения контрактов.

Описанная технология может использоваться для различных задач. Основными являются: использование ее для улучшения статического анализа, автоматическая или автоматизированная генерация документации к исходному коду.

Дальнейшее развитие результатов может выполняться в нескольких направлениях:
\begin{itemize}
\item развитие предложенной технологии с целью извлечения других видов контрактов (постусловий и инвариантов), а также с целью увеличения точности извлечения;
\item разработка более сложных алгоритмов слияния предикатов и рассмотрение других способов обработки противоположных предикатов;
\item изучение влияния извлеченных контрактов на статический анализ.
\end{itemize}
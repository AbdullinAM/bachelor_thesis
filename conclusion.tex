%%%%%%%%%%%%%%%%%%%%%%%%%%%%%%%%%%%%%%%%%%%%%%%%%%%%%%%%%%%%%%%%%%%%%%%%%%%%%%%%
\conclusion
%%%%%%%%%%%%%%%%%%%%%%%%%%%%%%%%%%%%%%%%%%%%%%%%%%%%%%%%%%%%%%%%%%%%%%%%%%%%%%%%
Одной из главных проблем статического анализа является необходимость повышения его полноты и точности. Основным способом решения данной проблемы является ручное задание пользовательских спецификаций к исходному коду. В данной работе рассматривается другой подход к решению данной проблемы: технология автоматического извлечения контрактов функций из исходного кода программ.

В работе выполнен анализ предметной области контрактного программирования, показано применение контрактов в современном мире. Так же в работе выполнен анализ исследований в области автоматического извлечения контрактов из исходного кода. Показана актуальность разработки технологии статического анализа программ с целью автоматического извлечения контрактов.

Основой предложенного подхода является анализ большого количества программных проектов с целью извлечения шаблонов вызовов функций в этих проектах. В работе представлен полный алгоритм автоматического извлечения контрактов из исходного кода.

На базе полученной технологии разработан прототип системы автоматического извлечения контрактов. Прототип состоит из двух частей: сервер БД и модуль автоматического извлечения контрактов для системы статического анализа Borealis. Разработанный прототип был протестирован на наборе реальных программных проектов различной сложности. Тестирование показало целесообразность применения технологии для автоматического извлечения контрактов.

Описанная технология может использоваться для различных задач. Основными являются: использование ее для улучшения статического анализа, автоматическая или автоматизированная генерация документации к исходному коду.

Дальнейшее развитие результатов может выполняться в нескольких направлениях. Возможно развитие предложенной технологии с целью извлечения других видов контрактов (постусловий и инвариантов) а так же с целью увеличения точности извлечения. Еще одним возможным направлением является изучение влияния разработанного алгоритма на статический анализ.
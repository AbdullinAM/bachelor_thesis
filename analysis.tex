%%%%%%%%%%%%%%%%%%%%%%%%%%%%%%%%%%%%%%%%%%%%%%%%%%%%%%%%%%%%%%%%%%%%%%%%%%%%%%%%
\chapter{АНАЛИЗ ИССЛЕДОВАНИЙ В ОБЛАСТИ АВТОМАТИЧЕСКОГО ИЗВЛЕЧЕНИЯ КОНТРАКТОВ ФУНКЦИЙ}
\label{chapter:analysis}
%%%%%%%%%%%%%%%%%%%%%%%%%%%%%%%%%%%%%%%%%%%%%%%%%%%%%%%%%%%%%%%%%%%%%%%%%%%%%%%%
В данном разделе рассматривается использование контрактов в статическом анализе. Проводится обзор способов задания пользовательских спецификаций, а также анализируются существующие в настоящий момент исследования в области автоматического извлечения контрактов.

%%%%%%%%%%%%%%%%%%%%%%%%%%%%%%%%%%%%%%%%%%%%%%%%%%%%%%%%%%%%%%%%%%%%%%%%%%%%%%%%
\section{Контракты в статическом анализе}
%%%%%%%%%%%%%%%%%%%%%%%%%%%%%%%%%%%%%%%%%%%%%%%%%%%%%%%%%%%%%%%%%%%%%%%%%%%%%%%%
Статический анализ кода --- анализ программного обеспечения, производимый (в отличие от динамического анализа) без реального выполнения исследуемых программ. Статические методы анализа ПО кроме исходного кода также используют дополнительные артефакты процесса разработки: объектные файлы, промежуточные представления, спецификации, схемы алгоритмов. Статические методы анализа становятся все более популярными в последнее время, так как развитие вычислительной техники позволяет снизить влияние одного из самых главных недостатков статических методов --- высокой вычислительной сложности.

Статические методы можно разделить на верификацию и статический анализ. Верификация --- это процесс математического доказательства соответствия программы исходным требованиям\footnote{Под верификацией здесь и далее понимается формальная верификация}. Преимущество верификации состоит в возможности проверки и доказательства корректности разработанной программы по отношению к совокупности формальных утверждений, представленных в спецификации. Главным ее недостатком является необходимость ручных подсказок о промежуточных и конечных целях доказательства и начальных условиях со стороны программиста. Эти условия, как правило, не могут быть построены автоматически и подчас требуют для своей формулировки не меньше усилий, чем разработка программы .

Статический анализ --- это анализ программы с целью выявления каких-либо ее свойств. Он ориентирован, как правило, на поиск нефункциональных ошибок. К основным достоинствам статического анали­за относится то, что он может быть полностью автоматическим, осуществляет поиск ошибок без участия пользователя, и, соответствен­но, практически лишён влияния человеческого фактора.  К недостаткам данной группы подходов относят большие вычислительные затраты, наличие ложных обнаружений, а также практически полное отсутствие возможности поиска функциональных ошибок.

Одним из способов борьбы с недостатками статического анализа является использование контрактов, позаимствованных из контрактного программирования. Контрактное программирование (design by contract, programming by contract, contract-based programming, DbC) --- это метод проектирования программного обеспечения. Он предполагает, что проектировщик должен определить формальные, точные и верифицируемые спецификации интерфейсов для компонентов системы. При этом, кроме обычного определения абстрактных типов данных, также используются контракты или утверждения(assertions). Термин был введен Бертраном Мейером при разработке языка Eiffel~\cite{eiffel} и был описан в работах~\cite{designByContract} и~\cite{oosc-meyer}.

Контракты --- это булевы выражения, описывающие состояние программы. Выделяют три основных типа контрактов: предусловия, постусловия и инварианты.
\begin{itemize}
\item Предусловия --- обязательства, которые любой клиентский модуль должен выполнить перед вызовом метода. Если данные обязательства не выполнены, то метод не может гарантировать корректность результатов его выполнения;

\item Постусловия --- определенные свойства, которые выполняются после выполнения метода. Выполнение этих условий является обязательствами модуля, содержащего соответствующий метод. Кроме того, при наличии постусловий этот модуль должен гарантировать завершаемость метода;

\item Инварианты --- свойства, которые должны выполняться как до, так и после вызова метода . Говоря об инвариантах обычно подразумевают инварианты класса --- глобальные свойства класса, определяющие семантические свойства и ограничения целостности, характеризующие его.
\end{itemize}

Существует множество инструментов статического анализа, поддерживающих контракты. Рассмотрим примеры подобных инструментов.

%%%%%%%%%%%%%%%%%%%%%%%%%%%%%%%%%%%%%%%%%%%%%%%%%%%%%%%%%%%%%%%%%%%%%%%%%%%%%%%%
\subsection{Библиотека CodeContracts}
%%%%%%%%%%%%%%%%%%%%%%%%%%%%%%%%%%%%%%%%%%%%%%%%%%%%%%%%%%%%%%%%%%%%%%%%%%%%%%%%
CodeContracts --- библиотека для платформы .NET от компании Microsoft, позволяющая задавать контракты в исходном коде и работать с предусловиями, постусловиями и инвариантами~\cite{codeContracts}. Пример приведен на листинге \ref{listing:codeContractsExample}. На основе данной библиотеки также разработано множество инструментов для автоматизации тестирования, автоматической генерации тестов, генерации документации, статического и динамического анализа, одним из которых является статический анализатор The Code Contracts static checker~\cite{cccheck}.
\lstinputlisting[
label={listing:codeContractsExample},
caption={Пример задания контрактов с помощью библиотеки CodeContracts},
]{src/codeContractsExample.cs}

%%%%%%%%%%%%%%%%%%%%%%%%%%%%%%%%%%%%%%%%%%%%%%%%%%%%%%%%%%%%%%%%%%%%%%%%%%%%%%%%
\subsection{Статический анализатор Frama-C}
%%%%%%%%%%%%%%%%%%%%%%%%%%%%%%%%%%%%%%%%%%%%%%%%%%%%%%%%%%%%%%%%%%%%%%%%%%%%%%%%
Еще одним популярным инструментом является статический анализатор Frama-C~\cite{framaC}. Инструмент объединяет в себе несколько подходов к статическому анализу, в том числе --- возможность задавать пользовательские спецификации на языке ACSL~\cite{acsl}. Этот язык позволяет задавать предусловия, постусловия и несколько видов инвариантов. Анализатор Frama-C затем проверяет выполнение заданных спецификаций.
\lstinputlisting[
label={listing:framaCExample},
caption={Пример задания спецификаций на языке ACSL},
]{src/framaCExample.c}


Два рассмотренных инструмента являются характерными представителями статических анализаторов. Стандартным способом задания контрактов для статических анализаторов (в том числе и для вышеописанных) является их ручное написание. При этом есть два разных способа: описание на уровне исходного кода или задание с помощью какого-либо языка аннотаций. Минусы ручного написания контрактов очевидны и полностью совпадают с минусами контрактного программирования --- ручное написание контрактов требует большого количества усилий и времени.

Одним из способов решения данной проблемы является автоматическое извлечение контрактов. Эта идея уже не нова, первые исследования в этой области начали проводиться в 2000-ых годах. Рассмотрим основные работы в этой области более подробно.

%%%%%%%%%%%%%%%%%%%%%%%%%%%%%%%%%%%%%%%%%%%%%%%%%%%%%%%%%%%%%%%%%%%%%%%%%%%%%%%%
\section{Автоматическое извлечение контрактов из исходного кода}
%%%%%%%%%%%%%%%%%%%%%%%%%%%%%%%%%%%%%%%%%%%%%%%%%%%%%%%%%%%%%%%%%%%%%%%%%%%%%%%%
Первые исследования в области автоматического извлечения контрактов проводились Николой Милановичем и Мирославом Малеком. В статье~\cite{extractingContractsFromJava} они рассматривают способы извлечения предусловий, постусловий и инвариантов из исходного кода Java. Основная идея --- анализ методов класса, их вызовов и условий выбрасывания исключений. Для извлечения контрактов используется гибридный метод, основанный на комбинации динамического и статического анализа. Также в данной работе рассматривается возможность извлечения контрактов из документации, однако проблема в том что для этого необходим формальный инструмент анализа документации, создать который практически невозможно.

Параллельно с ними свои исследования проводили Бертран Мейер и Карина Арнаут. В статье~\cite{uncoveringHiddenContracts} они на примере анализа классов стандартной библиотеки .NET рассмотрели способ извлечения контрактов на основе статического анализа метаданных компонентов .NET. Целью работы они ставят создание документированных оберток над стандартными классами .NET. Также в работе рассматриваются способы расширения методологии на другие языки. 

Автоматическое извлечение контрактов с помощью динамических методов впервые было рассмотрено в работах группы ученых из Университета Вашингтона: Майкла Эрнста, Джейка Кокрелла, Уильяма Грисуольда и Дэвида Ноткина. В 1999 году они опубликовали статью~\cite{discoveringInvariants}, в которой описали разработку инструмента Daikon. Описанный в статье подход основан на анализе трасс выполнения программы. Для получения трасс в подходе используется автогенерация тестов несколькими способами. Данный инструмент показал хорошие результаты и продолжил свое развитие. В настоящее время инструмент Daikon доступен для свободного использования~\cite{daikon}.

Примером более поздих работ в этой области является статья~\cite{staticPredicateMining}. В этой работе спецификации делятся на два типа: контракты и спецификации библиотек. Основная идея подхода: анализировать места вызова функций и пытаться выявлять повторяющиеся шаблоны. Результирующий шаблон вызова функции состоит из наиболее часто встречающихся перед соответствующим вызовом инструкций. Шаблоны могут включать в себя различные проверки условий, вызовы других функций и так далее. Повторяющиеся в шаблонах последовательности вызовов функций образуют спецификации библиотек. Остальные инструкции образуют предусловия функций. Прототип, реализующий предложенную методику, был протестирован на большом количестве программ на языке С и показал неплохие результаты.

Одним из последних исследований в этой области является работа~\cite{automiticAssertionsExtraction}. В данной работе описана разработка инструмента для автоматического извлечения темпоральных утверждений (temporal assertions) из трасс выполнения поведенческих моделей. Под темпоральными утверждениями при этом понимаются утверждения, являющиеся композицией обычных утверждений при помощи темпоральных операторов. Разработанный инструмент использует гибридный подход: для изначального извлечения утверждений используется рассмотренный ранее инструмент Daikon, затем извлеченные утверждения сортируются, объединяются и выражаются через темпоральные операторы с помощью различных статических методов.

На основе проведенного обзора можно сделать вывод, что большинство современных подходов являются динамическими или гибридными. Это имеет свои недостатки, так как для использования динамических методов необходимо иметь возможность запуска программы и получения ее трасс выполнения, что не всегда возможно. В то же время,  статические подходы можно расширить для получения более качественных результатов. Таким образом, разработка статического инструмента для автоматического извлечения контрактов из исходного кода является актуальной.

%%%%%%%%%%%%%%%%%%%%%%%%%%%%%%%%%%%%%%%%%%%%%%%%%%%%%%%%%%%%%%%%%%%%%%%%%%%%%%%%
\section{Резюме}
%%%%%%%%%%%%%%%%%%%%%%%%%%%%%%%%%%%%%%%%%%%%%%%%%%%%%%%%%%%%%%%%%%%%%%%%%%%%%%%%
В данном разделе рассмотрены способы использования контрактов при статическом анализе. Проанализированы существующие подходы к заданию контрактов для исходного кода. Проведен обзор существующих исследований в области автоматического извлечения контрактов. На основе проведенного обзора обоснована актуальность разработки статического метода автоматического извлечения контрактов.
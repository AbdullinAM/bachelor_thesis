%%%%%%%%%%%%%%%%%%%%%%%%%%%%%%%%%%%%%%%%%%%%%%%%%%%%%%%%%%%%%%%%%%%%%%%%%%%%%%%%
\chapter{Анализ исследований в области автоматического извлечения контрактов функций}
%%%%%%%%%%%%%%%%%%%%%%%%%%%%%%%%%%%%%%%%%%%%%%%%%%%%%%%%%%%%%%%%%%%%%%%%%%%%%%%%
В данном разделе проводится обзор контрактного программирования. Рассматривается использование контрактов в статическом анализе. Проводится обзор способов задания пользовательских спецификаций. Проводится анализ существующих в настоящий момент исследований в области автоматического извлечения контрактов.

%%%%%%%%%%%%%%%%%%%%%%%%%%%%%%%%%%%%%%%%%%%%%%%%%%%%%%%%%%%%%%%%%%%%%%%%%%%%%%%%
\section{Контрактное программирование}
%%%%%%%%%%%%%%%%%%%%%%%%%%%%%%%%%%%%%%%%%%%%%%%%%%%%%%%%%%%%%%%%%%%%%%%%%%%%%%%%
Контрактное программирование (design by contract, programming by contract, contract-based programming) --- это метод проектирования программного обеспечения. Он предполагает, что проектировщик должен определить формальные, точные и верифицируемые спецификации интерфейсов для компонентов системы. При этом, кроме обычного определения абстрактных типов данных, также используются предусловия, постусловия и инварианты. Данные спецификации называются «контрактами».

Термин был введен Бертраном Мейером в связи с разработкой языка Eiffel и был описан в статье <<Design by contract>>\cite{designByContract} и книге <<Object-Oriented Software Construction>>\cite{oosc-meyer}. DbC появилось вследствие работ по формальной верификации, формальной спецификации и логики Хоара.

Основная идея контрактного программирования --- это модель взаимодействия элементов программной системы, основывающаяся на идее взаимных обязательств и преимуществ. Как и в бизнесе, <<клиент>> и <<поставщик>> действуют в соответствии с определенным контрактом, который включает:
\begin{itemize}
\item обязательства, которые любой клиентский модуль должен выполнить перед вызовом вызовом метода --- так называемые предусловия. Эти предусловия дают преимущество поставщику --- он может не проверять выполнение этих предусловий;
\item определенные свойства, которые выполняются после выполнения метода --- постусловия. Выполнение этих условий является обязательствами поставщика;
\item обязательства по выполнению конкретных свойств --- инвариантов, которые должны выполняться и при вызове метода, и при выходе из него.
\end{itemize}

Контракты семантически эквивалентны тройке Хоара.

Многие языки программирования включают инструменты для проверки подобных требований.
%%%%%%%%%%%%%%%%%%%%%%%%%%%%%%%%%%%%%%%%%%%%%%%%%%%%%%%%%%%%%%%%%%%%%%%%%%%%%%%%
\chapter{Анализ исследований в области автоматического извлечения контрактов функций}
%%%%%%%%%%%%%%%%%%%%%%%%%%%%%%%%%%%%%%%%%%%%%%%%%%%%%%%%%%%%%%%%%%%%%%%%%%%%%%%%
В данном разделе проводится обзор контрактного программирования. Рассматривается использование контрактов в статическом анализе. Проводится обзор способов задания пользовательских спецификаций. Проводится анализ существующих в настоящий момент исследований в области автоматического извлечения контрактов.

%%%%%%%%%%%%%%%%%%%%%%%%%%%%%%%%%%%%%%%%%%%%%%%%%%%%%%%%%%%%%%%%%%%%%%%%%%%%%%%%
\section{Контрактное программирование}
%%%%%%%%%%%%%%%%%%%%%%%%%%%%%%%%%%%%%%%%%%%%%%%%%%%%%%%%%%%%%%%%%%%%%%%%%%%%%%%%
Контрактное программирование (design by contract, programming by contract, contract-based programming) --- это метод проектирования программного обеспечения. Он предполагает, что проектировщик должен определить формальные, точные и верифицируемые спецификации интерфейсов для компонентов системы. При этом, кроме обычного определения абстрактных типов данных, также используются <<контракты>> или утверждения(assertions).

Термин был введен Бертраном Мейером в связи с разработкой языка Eiffel и был описан в статье <<Design by contract>>\cite{designByContract} и книге <<Object-Oriented Software Construction>>\cite{oosc-meyer}. DbC появилось вследствие работ по формальной верификации, формальной спецификации и логики Хоара.

DbC является довольно простой, но, в то же время, мощной методикой, основанной на документировании прав и обязанностей программных модулей для обеспечения корректности программы. Основная идея контрактного программирования --- это модель взаимодействия элементов программной системы, основывающаяся на идее взаимных обязательств и преимуществ. Как и в бизнесе, <<клиент>> и <<поставщик>> действуют в соответствии с определенным контрактом. Контракты --- это булевы выражения, описывающие состояние программы. Выделяют три основных типа контрактов: предусловия, постусловия и инварианты.
\begin{itemize}
\item Предусловия --- обязательства, которые любой клиентский модуль должен выполнить перед вызовом вызовом метода. Если данные обязательства не выполнены, то метод не должен вызываться ни в коем случае. Эти предусловия дают преимущество поставщику --- он может не проверять выполнение этих предусловий;

\item Постусловия --- определенные свойства, которые выполняются после выполнения метода. Выполнение этих условий является обязательствами поставщика. Кроме того, наличие постусловий гарантирует завершение метода;

\item Инварианты --- свойства, которые должны выполняться и при вызове метода, и при выходе из него. Говоря об инвариантах обычно подразумевают инварианты класса --- глобальные свойства класса, определяющие более глубокие семантические свойства и ограничения целостности, характеризующие класс.
\end{itemize}

Многие языки программирования включают инструменты для задания подобных утверждений. Однако, DbC утверждает, что контракты являются ключевым инструментом создания корректного ПО, поэтому они должны быть утверждены на этапе проектирования. Таким образом, DbC предписывает начинать писать код с написания контрактов.

%%%%%%%%%%%%%%%%%%%%%%%%%%%%%%%%%%%%%%%%%%%%%%%%%%%%%%%%%%%%%%%%%%%%%%%%%%%%%%%%
\section{Контракты в статическом анализе}
%%%%%%%%%%%%%%%%%%%%%%%%%%%%%%%%%%%%%%%%%%%%%%%%%%%%%%%%%%%%%%%%%%%%%%%%%%%%%%%%
Статический анализ кода --- анализ программного обеспечения, производимый (в отличие от динамического анализа) без реального выполнения исследуемых программ. Статические методы анализа ПО очень часто кроме исходного кода так же используют дополнительные артефакты процесса компиляции: объектные файлы, промежуточные представления, спецификации, схемы алгоритмов. Статические методы анализа становятся все более популярными в наше время, так как развитие вычислительной техники позволяет снизить влияние одного из самых главных недостатков статических методов: высокой вычислительной сложности.

Статические методы можно разделить на методы верификации ПО и статический анализ ПО. Основным различием этих двух групп является то, что методы верификации основаны на применении математического аппарата для полного логического доказательства или опровержения каких-либо свойств кода, в то время как статический анализ представляет собой процесс аппроксимации кода, все доказываемые или опровергаемые им свойства носят вероятностный характер.

Главным достоинством верификации является ее математическая точность. Главным недостатком --- необходимость ручных подсказок о промежуточных и конечных целях доказательства и начальных условиях со стороны программиста. Эти условия, как правило, не могут быть построены ав­томатически и, подчас, требуют для своей формулировки не меньше усилий, чем сама программа.

К основным достоинствам статического анали­за относится то, что он может быть полностью автоматическим, осуществляет поиск ошибок без участия пользователя, и, соответствен­но, практически лишён влияния человеческого фактора. Кроме того, статический анализ ПО ориентирован, как правило, на поиск нефункциональных ошибок. К недостаткам данной группы подходов относят большие вычислительные затраты, наличие ложных обнаружений (в связи с аппроксимационной природой анализа), а также практически полное отсутствие возможности поиска функциональных ошибок.

Контрактное программирование как методика проектирования в современном мире используется достаточно редко. Исключениями являются только языки программирования, в которые DbC интегрирован изначально, и случаи, когда отсутствие ошибок в ПО является критическим. Основной причиной непопулярности DbC является необходимость ручного написания контрактов разработчиком. Зачастую это требует от него затрат большого количества времени и усилий. 

Однако, контракты часто применяются при статическом анализе и верификации как один из способов повышения полноты и точности. Для проведения анализа используют следующие критерии оценки корректности программных модулей:
\begin{itemize}
\item если инвариант класса и предусловия будут верны перед вызовом метода, то инвариант и постусловия будут верны после того, как метод завершит работу;

\item при вызове какого-либо метода программный модуль не должен нарушать его предусловия.
\end{itemize}

Существует множество инструментов статического анализа, поддерживающих контракты. В таблице \ref{tab::sai_with_contracts} приведены наиболее популярные из них.

\begin{table}
\caption{Инструменты статического анализа, поддерживающие использование контрактов}
	\begin{center}
		\begin{tabular}{|c|c|c|c|}
		\hline 
		Название инструмента & Компания & ЯП & Ссылка \\ 
		\hline 
		Code Contracts for .NET & Microsoft & .NET & \cite{cccheck} \\ 
		\hline 
		\end{tabular} 
	\end{center}
\label{tab::sai_with_contracts}
\end{table}
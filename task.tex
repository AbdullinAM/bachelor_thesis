%%%%%%%%%%%%%%%%%%%%%%%%%%%%%%%%%%%%%%%%%%%%%%%%%%%%%%%%%%%%%%%%%%%%%%%%%%%%%%%%
\chapter{ПОСТАНОВКА ЗАДАЧИ АВТОМАТИЧЕСКОГО ИЗВЛЕЧЕНИЯ КОНТРАКТОВ}
\label{chapter:task}
%%%%%%%%%%%%%%%%%%%%%%%%%%%%%%%%%%%%%%%%%%%%%%%%%%%%%%%%%%%%%%%%%%%%%%%%%%%%%%%%
Для успешной разработки технологии автоматического извлечения контрактов из исходного кода необходимо четко определить решаемую задачу, в том числе какие возможности должна включать в себя разрабатываемая технология. В данном разделе:
\begin{itemize}
\item выбирается анализируемый язык программирования;
\item определяется функциональность технологии автоматического извлечения контрактов;
\item предлагается подход к решению задачи автоматического извлечения контрактов;
\item определяются требования к прототипу системы автоматического извлечения контрактов.
\end{itemize}

%%%%%%%%%%%%%%%%%%%%%%%%%%%%%%%%%%%%%%%%%%%%%%%%%%%%%%%%%%%%%%%%%%%%%%%%%%%%%%%%
\section{Выбор целевого языка программирования}
%%%%%%%%%%%%%%%%%%%%%%%%%%%%%%%%%%%%%%%%%%%%%%%%%%%%%%%%%%%%%%%%%%%%%%%%%%%%%%%%
Одним из этапов выполнения работы является разработка модуля автоматического извлечения контрактов для системы статического анализа Borealis\cite{borealis}. Поэтому решено изначально в качестве целевого языка выбрать целевой язык анализатора --- С\cite{languageC}.

Задача автоматического извлечения контрактов для языка С является весьма сложной (как, впрочем, и любой другой его анализ). Одной из главных проблем являются указатели, которые приводят к необходимости выполнения анализа указателей (points-to analysis), так как через указатели возможно осуществление неявного взаимовлияния различных участков кода друг на друга. Кроме того в языке присутствует препроцессор, адресная арифметика, вычисляемые ссылки. Все эти конструкции оказывают сильное влияние на сложность анализа программы.

Ещё одним минусом данного языка с точки зрения извлечения контрактов является отсутствие каких-либо вспомогательных метаданных и механизма исключений. Например, в языке Java\cite{languageJava} существуют документирующие комментарии, анализируя которые можно значительно облегчить задачу извлечения контрактов. Также в Java присутствуют исключения, анализ которых тоже помогает при извлечении контрактов.

Вместе с тем язык С до сих пор является достаточно популярным. Многие современные инструменты и библиотеки разрабатываются на этом языке из-за его широких возможностей. Согласно данным на апрель 2016\footnote{http://www.tiobe.com/tiobe\_index} года язык С является вторым по популярности и занимает $13.9\%$ рынка. Полный рейтинг языков приведен в таблице \ref{table:languages}. На основе этого можно сказать, что анализ языка С с целью извлечения контрактов для улучшения статического анализа является актуальной задачей.

\begin{table}
	\caption{Рейтинг популярности языков программирования}
	\begin{center}
	\begin{tabular}{|l|l|l|}
	\hline 
	\textbf{Рейтинг} & \textbf{Язык} & \textbf{Процент}	\\ 
	\hline 
	1 & Java & 20.84\% \\ 
	\hline 
	2 & C & 13.9\% \\ 
	\hline 
	3 & C++ & 5.92\% \\ 
	\hline 
	4 & C\# & 3.79\% \\ 
	\hline 
	5 & Python & 3.33\% \\ 
	\hline 
	6 & PHP & 2.99\% \\ 
	\hline 
	7 & JavaScript & 2.56\% \\ 
	\hline 
	8 & Perl & 2.52\% \\ 
	\hline 
	9 & Ruby & 2.34\% \\ 
	\hline 
	10 & Visual Basic .NET & 2.27\% \\ 
	\hline 
	\end{tabular} 
	\end{center}
	\label{table:languages}
\end{table}

%%%%%%%%%%%%%%%%%%%%%%%%%%%%%%%%%%%%%%%%%%%%%%%%%%%%%%%%%%%%%%%%%%%%%%%%%%%%%%%%
\section{Задача автоматического извлечения контрактов функций}
%%%%%%%%%%%%%%%%%%%%%%%%%%%%%%%%%%%%%%%%%%%%%%%%%%%%%%%%%%%%%%%%%%%%%%%%%%%%%%%%
Задача данного исследования состоит в разработке подхода к автоматическому извлечению контрактов функций из исходного кода программ на основе анализа их  использования. Основным классом функций, для которых ожидается хороший результат, являются системные функции, так как они используются достаточно часто в большинстве проектов. При этом не исключается получение положительных результатов и для других классов функций. 

Предлагаемый подход позволяет извлекать предусловия вызова функций из исходного кода. Планируется некоторое расширение рассмотренного ранее статического подхода к извлечению контрактов\cite{staticPredicateMining}. Основные идеи предлагаемого подхода:
\begin{itemize}
\item анализ не всех инструкций, предшествующих вызову функции (как это было в ранее рассмотренном подходе), а только условных операторов. Это позволит облегчить анализ благодаря уменьшению объема анализируемой информации. Однако, анализ только условных операторов исключает возможность извлечения спецификаций библиотек;
\item анализ большого количества исходного кода из разных проектов с целью составления базы контрактов для большого количества функций;
\item составление контракта для функции путем объединения соответствующих данных из общей базы контрактов;
\end{itemize}

Полученные контракты затем можно использовать для решения нескольких задач. Главной задачей данной работы является автоматическое извлечение контрактов с целью их дальнейшего использования в статическом анализаторе для увеличения полноты и точности анализа. Однако, полученные результаты также можно использовать и для других целей: например, для автоматической генерации документации или помощи в проведении аудита программы. Предлагаемый подход разрабатывался для языка С, однако его без потери общности можно применить и к другим ЯП.

%%%%%%%%%%%%%%%%%%%%%%%%%%%%%%%%%%%%%%%%%%%%%%%%%%%%%%%%%%%%%%%%%%%%%%%%%%%%%%%%
\section{Задача разработки модуля автоматического извлечения контрактов для системы Borealis}
%%%%%%%%%%%%%%%%%%%%%%%%%%%%%%%%%%%%%%%%%%%%%%%%%%%%%%%%%%%%%%%%%%%%%%%%%%%%%%%%
Borealis --- это система статического анализа языка С, основанная на методе ограниченной проверки моделей (Bounded Model Checking, BMC)\cite{bmc}. Система построена на использовании компилятора Clang\cite{clang} для разбора исходного кода, системы LLVM\cite{llvm} для анализа кода и SMT решателей для поиска дефектов. Система способна искать дефекты двух типов: встроенные нефункциональные и заданные при помощи контрактов. Контракты задаются двумя способами: с помощью языка аннотаций, основанного на комментариях и схожего с языком ACSL, а также с помощью встроенных в программный код вызовов специальных процедур. В данной работе рассматривается третий способ задания контрактов для системы Borealis: автоматическое извлечение предусловий из исходного кода программ.

Модуль автоматического извлечения контрактов для системы Borealis должен обладать следующей функциональностью:
\begin{itemize}
\item анализ исходного кода программы с целью выявления условий, которые предположительно являются контрактами;
\item чтение/запись извлеченных предикатов из/в базу данных для сохранения результатов анализа;
\item формирование контрактов для функции из ранее определенного набора условий;
\item представление извлеченных контрактов в формате, понятном статическому анализатору.
\end{itemize}
Язык разработки прототипа --- С++\cite{languageC++}, так как система Borealis разрабатывается на языке С++.

%%%%%%%%%%%%%%%%%%%%%%%%%%%%%%%%%%%%%%%%%%%%%%%%%%%%%%%%%%%%%%%%%%%%%%%%%%%%%%%%
\section{Резюме}
%%%%%%%%%%%%%%%%%%%%%%%%%%%%%%%%%%%%%%%%%%%%%%%%%%%%%%%%%%%%%%%%%%%%%%%%%%%%%%%%
В данном разделе поставлена задача разработки технологии автоматического извлечения контрактов из исходного кода, определены основные требования. Также поставлена задача разработки прототипа автоматического извлечения контрактов для системы статического анализа Borealis для проверки работоспособности технологии и проведения качественной и количественной оценки эффективности алгоритма. 